

\documentclass[journal]{IEEEtran}

%%%%%%%%%%%%%%%%%%%%%%%%%%%%%%%%%%%%%%%%%%%%%%%%%%%%%%%%%%%%%%

\usepackage[italian]{babel}

% *** CITATION PACKAGES ***
\usepackage[style=ieee]{biblatex} 
\bibliography{digital.bib}    %your file created using JabRef
\usepackage{hyperref}

% *** MATH PACKAGES ***
\usepackage{amsmath}
 \usepackage{multirow}

% *** PDF, URL AND HYPERLINK PACKAGES ***
\usepackage{url}
% correct bad hyphenation here
\hyphenation{op-tical net-works semi-conduc-tor}
\usepackage{graphicx}  %needed to include png, eps figures
\usepackage{float}  % used to fix location of images i.e.\begin{figure}[H]

\usepackage{csquotes} % correzione errore compilazione

%%%%%%%%%%%%%%%%%%%%%%%%%%%%%%%%%%%%%%%%%%%%%%%%%%%%%%%%%%%%%


\begin{document}

% paper title
\title{Laboratorio di elettronica digitale\\ 
%\small{1 gennaio 2020}
}

% author names 
\author{\begin{center}Matteo Barbagiovanni\textsuperscript{1},
        Stefano Barbero\textsuperscript{2},
        Federico Malnati\textsuperscript{3},
        Valerio Pagliarino\textsuperscript{4},
        {\small \\
        \textsuperscript{1}
        matteo.barbagiovanni@edu.unito.it -
        \textsuperscript{2}
        stefano.barbero376@edu.unito.it
        \textsuperscript{3}
        federico.malnati@edu.unito.it -
        \textsuperscript{4}
        valerio.pagliarino@edu.unito.it}
        \end{center}}% <-this % stops a space
        
% The report headers
\markboth{Università degli Studi di Torino - C.d.L. Triennale in Fisica - 10/11/21 - A.A. 2021-2022    \quad   \quad \quad \quad   \quad \quad \quad  \quad   \quad \quad \quad   \quad \quad LABORATORIO DI ELETTRONICA \quad \quad }%do not delete next lines
{Shell \MakeLowercase{\textit{et al.}}: Bare Demo of IEEEtran.cls for IEEE Journals}

% make the title area
\maketitle


%%%%%%%%%%%%%%%%%%%%%%%%%%%%%%%%%%%%%%%%%%%%%%%%%%%%%%%%%%%%%
%% Introduzione 
%%%%%%%%%%%%%%%%%%%%%%%%%%%%%%%%%%%%%%%%%%%%%%%%%%%%%%%%%%%%%

\begin{abstract} 
Introduzione all'esperienza
\end{abstract}

%%%%%%%%%%%%%%%%%%%%%%%%%%%%%%%%%%%%%%%%%%%%%%%%%%%%%%%%%%%%%
%%%%%%%%%%%%%%%%%%%%%%%%%%%%%%%%%%%%%%%%%%%%%%%%%%%%%%%%%%%%%

\section{Componenti utilizzati}

Qui descrizione degli integrati logici utilizzati, direi che è un po' brutto fare dei copia e incolla della descrizione degli strumenti di misura che abbiamo già messo in analogica, siccome sia Aglietta che Migliore hanno letto quella relazione. 


\subsection{Logica TTL serie 7400}

\begin{figure}[H]%[!ht]
\begin{center}
\includegraphics[width=0.40\textwidth]{lab-reports/Schematics-and-graphics/SN7400N.png}
\caption{SN7400N}
\label{fig:integrated_nand}
\end{center}
\end{figure}

\subsection{Transistor di interfaccia PN2222A}






Questi valori li riporterei nel corpo della descrizione dello schema elettrico dell'ADC.
$R_b = 0.99 \pm 0.01 \ k\Omega \\$
$R_c = 99 \pm 1 \ k\Omega$




%%%%%%%%%%%%%%%%%%%%%%%%%%%%%%%%%%%%%%%%%%%%%%%%%%%%%%%%%%%%%
%% Primo Giorno 
%%%%%%%%%%%%%%%%%%%%%%%%%%%%%%%%%%%%%%%%%%%%%%%%%%%%%%%%%%%%%

\section{Tabella di verità R-S flip flop}

\begin{figure}[H]%[!ht]
\begin{center}
\includegraphics[width=0.40\textwidth]{sch-simulations/digital/output/flip-flop-RS.pdf}
\caption{Flip-flop set - reset}
\label{fig:circuit_flip_flop}
\end{center}
\end{figure}

%%%%%%%%%%%%%%%%%%%%%%%%%%%%%%%%%%%%%%%%%%%%%%%%%%%%%%%%%%%%%

\begin{figure}[H]%[!ht]
\begin{center}
\includegraphics[width=0.54\textwidth]{sch-simulations/digital/output/shift-register.pdf}
\caption{Registro a scorrimento e circuito di controllo}
\label{fig:circuit_shift_register}
\end{center}
\end{figure}


%%%%%%%%%%%%%%%%%%%%%%%%%%%%%%%%%%%%%%%%%%%%%%%%%%%%%%%%%%%%%
%% Secondo Giorno 
%%%%%%%%%%%%%%%%%%%%%%%%%%%%%%%%%%%%%%%%%%%%%%%%%%%%%%%%%%%%%

\section{Tabella di verità AND}

\subsection{Misura $V_{IH}$ }

%%%%%%%%%%%%%%%%%%%%%%%%%%%%%%%%%%%%%%%%%%%%%%%%%%%%%%%%%%%%%

\section{Tabella di verità JK}

%%%%%%%%%%%%%%%%%%%%%%%%%%%%%%%%%%%%%%%%%%%%%%%%%%%%%%%%%%%%%
%% Terzo Giorno 
%%%%%%%%%%%%%%%%%%%%%%%%%%%%%%%%%%%%%%%%%%%%%%%%%%%%%%%%%%%%%

\begin{figure*}[t]%[t]
\centering
\begin{center}
\includegraphics[width=1.15\textwidth, angle=90]{sch-simulations/digital/output/SAR_4bit_cv1.pdf}
\end{center}
\caption{Schema elettrico completo del convertitore ADC SAR 4 bit}
\label{fig:circuit_sarCompleteSchematic}
\end{figure*}



%%%%%%%%%%%%%%%%%%%%%%%%%%%%%%%%%%%%%%%%%%%%%%%%%%%%%%%%%%%%%
%% Appendice
%%%%%%%%%%%%%%%%%%%%%%%%%%%%%%%%%%%%%%%%%%%%%%%%%%%%%%%%%%%%%


\begin{appendices}

\section{Oscillatore ad anello}

\begin{figure}[H]%[!ht]
\begin{center}
\includegraphics[width=0.40\textwidth]{sch-simulations/digital/output/ring-osc-logic.pdf}
\caption{Circuito equivalente dell'oscillatore ad anello}
\label{fig:circuit_ring_oscillator}
\end{center}
\end{figure}

\begin{figure}[H]%[!ht]
\begin{center}
\includegraphics[width=0.48\textwidth]{analysis/output/inverter_ring_xy.pdf}
\caption{Studio dell'oscillatore ad anello}
\label{fig:inverter_ring_xy}
\end{center}
\end{figure}

%%%%%%%%%%%%%%%%%%%%%%%%%%%%%%%%%%%%%%%%%%%%%%%%%%%%%%%%%%%%%

\section{Caratterizzazione transistor npn}


\end{appendices}

%%%%%%%%%%%%%%%%%%%%%%%%%%%%%%%%%%%%%%%%%%%%%%%%%%%%%%%%%%%%%
%% Indice e Bibliografia 
%%%%%%%%%%%%%%%%%%%%%%%%%%%%%%%%%%%%%%%%%%%%%%%%%%%%%%%%%%%%%

\clearpage
\newpage

\tableofcontents % Indice

\newpage

\printbibliography % Bibliografia

\end{document}